\chapter{Introduction}\label{chapter:Introduction}
% \pagenumbering{arabic}

\indent The advent of the internet and digital technologies has revolutionized the way we access information and make decisions. One area where this impact is significantly felt is in the domain of food choices. With a plethora of options available, making a healthy and personalized food choice can be overwhelming. This is where food recommender systems come into play. They aim to provide personalized food recommendations that align with the user's preferences, dietary needs, and health goals.
\\

\indent However, existing food recommender systems have several limitations. They often ignore crucial factors such as the ingredients of the food, the time factor (evolution and change in user tastes and diets over time), cold start users and foods (which have insufficient ratings or interactions), and user communities (which can improve the quality and diversity of recommendations). These limitations can lead to recommendations that are not truly personalized or beneficial to the user.
\\

Food recommender systems are categorized into three types:
\begin{itemize}
  \item Collaboratve Filtering - where items are suggested based on the preferences and behavior of similar users. Yummly is an example of this, where recipes liked by users with similar tastes are recommended.
  \item Content-based Filtering- where items are suggested after their attributes are analyzed
and matched with the user's past interaction. Foodpairing is an example, where recommendation is done based on the flavor profile of ingredients.
  \item Hybrid model - which combines the above two methods for providing a more accurate and diverse recommendations. Examples are OpenTable and HelloFresh.
\end{itemize}

\indent TDLGC is a hybrid food recommender system that combines collaborative filtering and content-based filtering approaches to address the limitations of existing systems.