% Abstract of Seminar
\section*{\centering ABSTRACT}
\indent Recommender systems are important tools that deliver personalized content, and help recommend relevant products to customers in various domains including e-commerce, social media, and entertainment. Food recommendation systems provide an effective tool that can help users adjust their eating habits and provide a healthier diet.
\\
\indent Previous food recommender systems have shown to be effective in learning a user’s pattern by mapping previous interactions with food, but they still suffer some drawbacks, including: ingredients of food, time factor, cold start users and cold start foods, user’s community.
\\
\indent The system plans to overcome all these using a method called as “Time-aware food recommender-system based on Deep Learning and Graph Clustering [TDLGC]”\cite*{9775081}. It involves two phases: \\(i) recommendation based on users, and (ii) recommendation based on contents of food. First phase considers a similarity-matrix of the user to predict ratings and the second phase utilizes a deep-learning based clustering algorithm to group the foods into several clusters and then rating of unseen foods is predicted.
\\
\indent This  model has several novelties, such as: (i) ingredients-aware food recommender system, (ii) time-aware food recommender system, (iii) trust-aware food recommender system, (iv) community-aware food recommender system. Graph clustering is used to group nodes in a graph into clusters based on their similarity and connectivity. Edge weights are calculated using the similarities between user ratings. Deep learning uses a neural network in-order to learn the features of the users and food items. The time factor in this system is used to take into account the changes in food preferences over time.