\chapter{Conclusion} \label{chapter:Conclusion}

\indent The development and implementation of a robust and efficient food recommender system is a timely and relevant endavour in today's world. This is primarily due to the increasing health consciousness among people. As individuals strive for personalized dietary recommendations that align with their health goals, dietary restrictions and taste preferences, a system that can properly cater to these individuals is much needed.
\\
\indent The digital age has ushered in an era of information overload. The vast amount of food-related information widely available online can be overwhelming for many users. In such scenarios, recommender systems work as a valuable tool in helping the users to navigate through this plethora of information and discover new tools which align with their needs and goals.
\\
\indent Personalization is an important aspect which underscores the need for such a system. They significantly help in enhancing user satsifaction and engagement with the platform. This increases user retention and makes the system very beneficial for users.
\\
\indent In conclusion, TDLGC addresses several challenges faced by currently existing systems and meet the current needs and expectations of users for personalized and health-conscious food recommendations. It can greatly contribute to promotion of healthy eating habits of its users, thereby making it a valuable tool for the users.